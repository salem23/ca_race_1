%%%%%%%%%%%%%%%%%%%%%%%%%%%%%%%%%%%%%%%%%%%%%%%%%%%%%%%%%%%%%%%%%%%%%%%%%%%%%
% 26/05/2010
% edited by Bill Lampos
%
% Feel free to use (copy) the structure (latex formatting source code)
% but not the content of this document.
%
%%%%%%%%%%%%%%%%%%%%%%%%%%%%%%%%%%%%%%%%%%%%%%%%%%%%%%%%%%%%%%%%%%%%%%%%%%%%%

\PassOptionsToPackage{table}{xcolor}
\documentclass[compress]{beamer}
\mode<presentation>
\usepackage{etex}
\usetheme{Madrid}
% other themes: AnnArbor, Antibes, Bergen, Berkeley, Berlin, Boadilla, boxes, CambridgeUS, Copenhagen, Darmstadt, default, Dresden, Frankfurt, Goettingen,Warsaw   Berkeley Marburg
% Hannover, Ilmenau, JuanLesPins, Luebeck, Madrid, Maloe, Marburg, Montpellier, PaloAlto, Pittsburg, Rochester, Singapore, Szeged, classic

\setbeamertemplate{footline}[page number]{}

%\makeatother
%\setbeamertemplate{footline}
%{
%	\leavevmode%
%	\hbox{%
%		\begin{beamercolorbox}[wd=.4\paperwidth,ht=2.25ex,dp=1ex,center]{author in head/foot}%
%			\usebeamerfont{author in head/foot}\insertshortauthor
%		\end{beamercolorbox}%
%		\begin{beamercolorbox}[wd=.6\paperwidth,ht=2.25ex,dp=1ex,center]{title in head/foot}%
%			\usebeamerfont{title in head/foot}\insertshorttitle\hspace*{3em}
%			\insertframenumber{} / \inserttotalframenumber\hspace*{1ex}
%		\end{beamercolorbox}}%
%		\vskip0pt%
%	}
%	\makeatletter
\setbeamertemplate{navigation symbols}{}

\usecolortheme{sidebartab}
% color themes: albatross, beaver, beetle, crane, default, dolphin, dov, fly, lily, orchid, rose, seagull, seahorse, sidebartab, structure, whale, wolverine

\usefonttheme{serif}
% font themes: default, professionalfonts, serif, structurebold, structureitalicserif, structuresmallcapsserif

% pdf is displayed in full screen mode automatically
\hypersetup{pdfpagemode=FullScreen}

% define your own colours:
\definecolor{Red}{rgb}{1,0,0}
\definecolor{Blue}{rgb}{0,0,1}
\definecolor{Green}{rgb}{0,1,0}
\definecolor{magenta}{rgb}{1,0,.6}
\definecolor{lightblue}{rgb}{0,.5,1}
\definecolor{lightpurple}{rgb}{.6,.4,1}
\definecolor{gold}{rgb}{.6,.5,0}
\definecolor{orange}{rgb}{1,0.4,0}
\definecolor{hotpink}{rgb}{1,0,0.5}
\definecolor{newcolor2}{rgb}{.5,.3,.5}
\definecolor{newcolor}{rgb}{0,.3,1}
\definecolor{newcolor3}{rgb}{1,0,.35}
\definecolor{darkgreen1}{rgb}{0, .35, 0}
\definecolor{darkgreen}{rgb}{0, .6, 0}
\definecolor{darkred}{rgb}{.75,0,0}
\definecolor{LightCyan}{rgb}{0.88,1,1}
\xdefinecolor{olive}{cmyk}{0.64,0,0.95,0.4}
\xdefinecolor{purpleish}{cmyk}{0.75,0.75,0,0}


% \usepackage{beamerinnertheme_______}
% inner themes include circles, default, inmargin, rectangles, rounded

%\usepackage{beamerouterthemesmoothbars}
% outer themes include default, infolines, miniframes, shadow, sidebar, smoothbars, smoothtree, split, tree

\useoutertheme[subsection=false]{smoothbars}

%\setbeamercolor{frametitle}{fg=magenta,bg=yellow!30}
%\setbeamercolor{section in head/foot}{bg=blue!60}
%\setbeamercolor{author in head/foot}{bg=Green}
%\setbeamercolor{date in head/foot}{fg=Brown}



% to have the same footer on all slides
%\setbeamertemplate{footline}[text line]{xxx xxx xxx}
%\setbeamertemplate{footline}[text line]{} % or empty footer
%\usepackage[utf8]{inputenc}
%\usepackage[T1]{fontenc}
% include packages
\usepackage{subfigure}
\usepackage{multicol}
\usepackage{amsmath}
\usepackage{epsfig}
\usepackage{graphicx}
\usepackage[all,knot]{xy}
\xyoption{arc}
\usepackage{url}
\usepackage{multimedia}
\usepackage{hyperref}
\usepackage{setspace}
\usepackage{tikz}
\usepackage{pgfplots}
\usepackage{epsfig}
\usepackage{epstopdf}
%\usepackage{adjustbox}
\usepackage{booktabs}
\usepackage{anyfontsize}
\usepackage{geometry}
\usepackage{multirow}
\usepackage{slashbox}

\usepackage{xcolor,colortbl}



\usetikzlibrary{mindmap,shadows}
\usetikzlibrary{shapes,arrows}
\usetikzlibrary{calc,patterns,snakes,decorations.pathmorphing,decorations.markings}
\usetikzlibrary{positioning}

%\usetikzlibrary{shapes.geometric,backgrounds,positioning-plus,node-families}
\usetikzlibrary{plotmarks}

\tikzset{
	invisible/.style={opacity=0,text opacity=0},
	visible on/.style={alt=#1{}{invisible}},
	alt/.code args={<#1>#2#3}{%
		\alt<#1>{\pgfkeysalso{#2}}{\pgfkeysalso{#3}} % \pgfkeysalso doesn't change the path
	},
}

\usepackage{empheq}
\usepackage[table]{xcolor}
\usepackage[skins]{tcolorbox}

\setbeamercolor{block title}{bg=red!30,fg=blue}%bg=background, fg= foreground
\setbeamercolor{block body}{bg=green!40,fg=black}%bg=background, fg= foreground

\setbeamerfont{section number projected}{%
	family=\rmfamily,series=\bfseries,size=\normalsize}
\setbeamercolor{section number projected}{bg=black,fg=yellow}

\setbeamercolor{section in toc}{fg=red}
%\setbeamercolor{section in toc}{fg=blue}


\setbeamercolor{navigation symbols dimmed}{fg=red!80!black}
\setbeamercolor{navigation symbols}{fg=red!80!black}

%\institute{{\tiny \sffamily Soutenance de Doctorat en sciences}}
\title{\textcolor{yellow!50!}{ \bf{\tt }}}
%\subtitle{Tracking trends on the web using novel Machine Learning methods}


\author{{SALEM Mohammed } \hfill {  }}
%\institute{\tiny 
%		}
\date{\scriptsize D?partement d'Informatique\\ 05 F?vrier 2015}
\newcommand\Section[2][]{%
	\section<presentation>[#1]{#2}
	\section<article>{#2}
}

%\setbeamercolor*{section in head/foot}{fg=white,bg=black}

\newlength\figureheight 
\newlength\figurewidth  
\begin{document}
	
	
	\frame[plain]{
		\begin{center}
			%\includegraphics[width=12cm,height=2cm]{figures/univ2} \\ 
		\end{center}
		\begin{center} \sffamily Car race controller based on evolutionnary trained radial basis function neural networks \end{center}
		\begin{center} \textbf{March, 05, 2015} \end{center}
		

	}


\frame{\frametitle{Aims}
We will presents a controller for the 
Simulated Car Racing. 

We will start by training a neural based controller to drive alone in the track ( time trial).

The second step is to learn to compete with other ddrivers.

}
\frame{\frametitle{Description of the available sensors}
\scalebox{0.6}{
				\begin{tabular}{|c||l|m{12cm}|}
					\rowcolor{green!20}
					\hline
Name& Range(unit) & Description\\
angle& $[-\pi,+\pi]$ (rad)&
Angle between the car direction and the direction of the
track axis.\\
curLapTime &$[0,+\infty]$ (s) & Time elapsed during current lap.\\
damage& $[0,+\infty]$ (point)&
Current damage of the car (the higher is the value the higher
is the damage).\\
distFromStart& $[0,+\infty]$ (m) & Distance of the car from the start line along the track line.\\
distRaced &$[0,+\infty]$ (m)& Distance covered by the car from the beginning of the race\\

fuel& $[0,+\infty]$ (l) & Current fuel level.\\
gear &-1,0,1,0..6 g & Current gear: -1 is reverse, 0 is neutral and the gear from
1 to 6.\\
lastLapTime &$[0,+\infty]$ (s)& Time to complete the last lap\\
opponents&[0,200] (m)&
Vector of 36 opponent sensors: each sensor covers a span
of 10 degrees within a range of 200 meters and returns the
distance of the closest opponent in the covered area. \\
racePos & 1,2,..N&  Position in the race with respect to other cars.\\
rpm & $[0,+\infty]$ (rpm) & Number of rotation per minute of the car engine.\\
speedX & $[-\infty,+\infty]$  (km/h)& Speed of the car along the longitudinal axis of the car.\\
speedY & $[-\infty,+\infty]$  (km/h)& Speed of the car along the transverse axis of the car.\\
speedZ & $[-\infty,+\infty]$  (km/h)& Speed of the car along the Z axis of the car.\\
	
track & [0,200] (m)& 
Vector of 19 range finder sensors: each sensors returns the
distance between the track edge and the car within a range
of 200 meters.\\	
trackPos & $[-\infty,+\infty]$& 
Distance between the car and the track axis. The value is
normalized w.r.t to the track width: it is 0 when car is on
the axis, -1 when the car is on the right edge of the track
and +1 when it is on the left edge of the car. Values greater
than 1 or smaller than -1 mean that the car is outside of
the track.\\	
%wheelSpinVel&  [0,+1] (rad/s)& 
%Vector of 4 sensors representing the rotation speed of
%wheels.\\	
%z&  $[-\infty,+\infty]$ (m)&
%Distance of the car mass center from the surface of the track
%along the Z axis.\\			
			
				\hline		
			
		\end{tabular}
}
}


	\frame{\frametitle{Description of the available effectors}
		\scalebox{0.9}{
			\begin{tabular}{|c||l|m{9cm}|}
				\rowcolor{green!20}
				\hline
				Name& Range & Description\\
accel& [0,1] &Virtual gas pedal (0 means no gas, 1 full gas).\\
brake &[0,1]& Virtual brake pedal (0 means no brake, 1 full brake).\\
clutch &[0,1]& Virtual clutch pedal (0 means no clutch, 1 full clutch).\\
gear &-1,0,1,0..6& Gear value.\\
steering &[-1,1]&
Steering value: -1 and +1 means respectively full right and
left, that corresponds to an angle of 0.366519 rad.\\
focus &[-90,90] &Focus direction in degrees.\\
meta & 0,1&
This is meta-control command: 0 do nothing, 1 ask competition server to restart the race.\\
				
				\hline		
				
			\end{tabular}
		}
	}


	\frame{\frametitle{Designing TORCS AI based controller}	
		
		\begin{itemize}
			\item High level control\\
			use information from the sensors to predict the trajectory in the next segment of the track ( with scripted driving policy to compute next gear , accel cluch and steering). This is could be done either by collecting data from the warm lap with low speed or adding a new sensor to get ahead information about the track as in \\
	\textbf	{L. Cardamone, D. Loiacono, and P. Lanzi, ?Learning drivers for
			TORCS through imitation using supervised methods,? in Proceedings
			of the 5th international conference on Computational Intelligence and
			Games. IEEE Press, 2009, pp. 148?155.	}
			\item Low level control : predict actions of the car directly(ifficult in case of noise).
			
		\end{itemize}
	}	





	\frame{\frametitle{Designing indirect human-like controller}	

	\begin{columns}[T] % align columns
		\begin{column}{.49\textwidth}
			%	\color{red}\rule{\linewidth}{4pt}
			\begin{block}{ RBF neural network Inputs:}
\begin{itemize}
	\item angle 
	\item gear
	\item distRaced
	\item lastLapTime	
	\item speedX
	\item speedY
	\item  focus
\end{itemize}
			\end{block}
		\end{column}%
		\hfill%
		\begin{column}{.49\textwidth}
			%	\color{blue}\rule{\linewidth}{4pt}
			
			\begin{block}{ RBF neural network Outputs}
\begin{itemize}
	\item targetspeed 
	\item targetposition
\end{itemize}
				
			\end{block}
		\end{column}%
	\end{columns}
	\vspace{0.5cm}

}


	\frame{\frametitle{Designing indirect human-like controller}	

\begin{itemize}
\item Train the RBFNN with GA.
\item Hidden neurons number: Trial /error.\\
\item  Driving policy is used to compute the desired steer, gear acel and cluch to move the car from the currentpos and currspeed to target position and speed.\\
\item  $Fitness= \sum{(targetpos-currpos)}^2$

\item  Another way is to collect data from some tracks in the TORCS and train the RBFNN with BP or another training algorithm.	
\end{itemize}


}	
	
	
	
		\frame{\frametitle{Designing direct human-like controller}	
			
			\begin{columns}[T] % align columns
				\begin{column}{.49\textwidth}
					%	\color{red}\rule{\linewidth}{4pt}
					\begin{block}{ RBF neural network Inputs:}
						\begin{itemize}
							\item angle 
							\item focus 
							\item damage
							\item distRaced
							\item fuel 
							\item gear 
							\item lastLapTime	
							\item rpm
							\item speedX
							\item speedY
							\item track
							\item wheelSpinVel
							\item  focus
						\end{itemize}
					\end{block}
				\end{column}%
				\hfill%
				\begin{column}{.49\textwidth}
					%	\color{blue}\rule{\linewidth}{4pt}
					
					\begin{block}{ RBF neural network Outputs}
						\begin{itemize}
							\item accel 
							\item brake 
							\item clutch 
							\item gear 
							\item steering 
							\item focus	
						\end{itemize}
						
					\end{block}
				\end{column}%
			\end{columns}
			
		}
		
		
		\frame{\frametitle{Designing direct human-like controller}	
			
			\begin{itemize}
				\item Hidden neurons number: Trial /error.\\
				\item Train the RBFNN by a genetic algorithm.\\
				\item Chromosome=[angle, currlapt, distFromStart,distRaced, fuel, gear, lastLapTime, rpm ]\\
				
			\end{itemize}
			
			\begin{enumerate}
				\item Fitness1:
				\begin{equation}
				f_{g}= a*damage+b*fuel+c*lastLapTime+d*currlapt+e*angle
				\end{equation}
				\item  Fitness2:
				\begin{equation}
				f_{g}= (1-e^{\beta})(currlapt+angle)+e^{\beta}(damage+fuel)
				\end{equation}
				\item  
				Fitness3:
				a fitness for each 20ms of the race ( well position the car in the track)
				and a global function ( sum of ladtlaps time or points accorded to the controller)
				\begin{equation}
				f_{g}= (1-e^{\beta})(currlapt+angle)+e^{\beta}(damage+fuel)
				\end{equation}
			\end{enumerate}	
		}	
		
\end{document} 